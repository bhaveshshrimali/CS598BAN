\documentclass{beamer}
\usetheme{CambridgeUS}
\usepackage{amsmath, amsfonts, amssymb, amsthm}
\usepackage{bm, enumitem, booktabs}
\usepackage{mathtools}
\usepackage{tensor}
%Information to be included in the title page:
% \def\Ktan{\mathcal{K}}
% \def\Jtan{\mathcal{J}}
\def\myPhi{{\mathrm{\Phi}}}
\def\myPsi{{\mathrm{\Psi}}}
\def\myF{{\mathsf{F}}}
\def\myS{\bf\mathsf{S}}
\def\myC{{\bf\mathsf{C}}}
\def\myT{{\mathsf{T}}}
\def\myM{{\mathsf{M}}}
\newcommand\lrB[1]{{\left[#1 \right]}}
%SYMBOL DEFINITIONS
%
%
%===================================================================
% Boldface letters
%===================================================================
\def\bfa{{\bf a}}
\def\bfn{{\bf n}}
\def\bfm{{\bf m}}
\def\bfl{{\bf l}}
\def\bfu{{\bf u}}
\def\bfv{{\bf v}}
\def\bfx{{\bf x}}
\def\bfy{{\bf y}}
\def\bfz{{\bf z}}
\def\bfA{{\bf A}}
\def\bfE{{\bf E}}
\def\bfB{{\bf B}}
\def\bfC{{\bf C}}
\def\bfH{{\bf H}}
\def\bfI{{\bf I}}
\def\bfK{{\bf K}}
\def\bfL{{\bf L}}
\def\bfM{{\bf M}}
\def\bfN{{\bf N}}
\def\bfP{{\bf P}}
\def\bfQ{{\bf Q}}
\def\bfR{{\bf R}}
\def\bfS{{\bf S}}
\def\bfT{{\bf T}}
\def\bfX{{\bf X}}
\def\bfF{{\bf F}}
\def\bfU{{\bf U}}
\def\bfW{{\bf W}}
\def\bfY{{\bf Y}}
\def\bfZ{{\bf Z}}
%========================================================
% Greek bold face letters
%========================================================
\def\bfal{\mbox{\boldmath $\alpha$}}
\def\bfbe{\mbox{\boldmath $\beta$}}
\def\bfzeta{\mbox{\boldmath $\zeta$}}
\def\bfeps{\mbox{\boldmath $\epsilon$}}
\def\bfsig{\mbox{\boldmath $\sigma$}}
\def\bfD{{\bf D}}
\def\bfe{{\bf e}}
\def\bfg{\mbox{\boldmath $\gamma$}}
\def\bfG{\mbox{\boldmath $\Gamma$}}
\def\bfmu{\mbox{\boldmath $\mu$}}
\def\bfrho{{\bf S}}
\def\bfeta{\mbox{\boldmath $\eta$}}
\def\bfs{{\bf S}}
\def\bfXi{\mbox{\boldmath $\Xi$}}
\def\bfw{\mbox{\boldmath $\omega$}}
\def\bftau{\mbox{\boldmath $\tau$}}
\def\bfchi{\mbox{\boldmath $\chi$}}
\def\bfxi{\mbox{\boldmath $\xi$}}
\def\bfxip{\bfxi^{\perp}}
\def\bfG{\mbox{\boldmath $\Gamma$}}
\def\bfOm{\mbox{\boldmath $\Omega$}}
\def\Leff{\widetilde{\mbox{\boldmath$\mathcal{L}$}}}
\def\Ltan{\mbox{\boldmath$\mathcal{L}$}}
\def\Ktan{\mbox{\boldmath$\mathcal{K}$}}
\def\Jtan{\mbox{\boldmath$\mathcal{J}$}}
\def\Ctan{\mbox{\boldmath$\mathcal{C}$}}
\def\Mtan{\mbox{\boldmath$\mathcal{M}$}}
\def\Ptan{\mbox{\boldmath$\mathcal{P}$}}
\def\Qtan{\mbox{\boldmath$\mathcal{Q}$}}
\def\Stan{\mbox{\boldmath$\mathcal{S}$}}
\def\Atan{\mbox{\boldmath$\mathcal{A}$}}
\def\Itan{\mbox{\boldmath$\mathcal{I}$}}
\def\Htan{\mbox{\boldmath$\mathcal{H}$}}
\def\ehat{\hat{\bf e}}
%========================================================
% Greek letters
%========================================================
\def\a{\alpha}
\def\b{\beta}
\def\eps{\varepsilon}
\def\g{\gamma}
\def\l{\lambda}
\def\m{\mu}
\def\sig{\sigma}
\def\otheta{\overline{\theta}}
\def\oPsi{\overline{\Psi}}
\def\hPsi{\hat{\Psi}}
%========================================================
% Abbreviated forms
%========================================================
\def\eeq{\varepsilon_{e}}
\def\eem{\varepsilon_{m}}
\def\e0{\varepsilon_0}
\def\oQ{\overline{Q}}
\def\oR{\overline{R}}
\def\oS{\overline{S}}
\def\obfH{\overline{\bfH}}
\def\obfe{\overline{\bfeps}}
\def\obfeps{\overline{\bfeps}}
\def\bF{\overline{F}}
\def\oI{\overline{I}}
\def\oJ{\overline{J}}
\def\obfg{\overline{\bfg}}
\def\obfs{\overline{\bfs}}
\def\obfB{\overline{\bfB}}
\def\obfC{\overline{\bfC}}
\def\obfc{\overline{\bfc}}
\def\obfD{\overline{\bfD}}
\def\obfF{\overline{\bfF}}
\def\obfE{\overline{\bfE}}
\def\obfU{\overline{\bfU}}
\def\obfQ{\overline{\bfQ}}
\def\obfK{\overline{\bfK}}
\def\obfR{\overline{\bfR}}
\def\obfS{\overline{\bfS}}
\def\ol{\overline{\lambda}}
\def\oe{\overline{\varepsilon}}
\def\oos{\overline{\overline{\sigma}}}
\def\p{\partial}
\def\og{\overline{g}}
\def\oh{\overline{h}}
\def\oJ{\overline{J}}
\def\seq{\sigma_{eq}}
\def\sm{\sigma_{m}}
\def\s0{\sigma_0}
\def\v{\varphi}
\def\nn{\nonumber}
\def\Weff{\overline{W}}
\def\Pheff{\overline{\Phi}}
\def\meff{\overline{\mu}}
\def\keff{\overline{\kappa}}
\def\oL{\overline{\Lambda}}

\def\oH{\overline{H}}
\def\oF{\overline{F}}
\def\oL{\overline{L}}
\def\oLr{\overline{\Lr}}
\def\bfLr{{\bf \Lr}}
\def\bfwN{\bfw\otimes\bfxi}
\def\oJ{\overline{J}}

\def\obfS{\overline{\bfS}}
\def\obfF{\overline{\bfF}}
\def\obfH{\overline{\bfH}}
\def\obfL{\overline{\bfL}}
\def\obfLr{\overline{\bfLr}}
\def\cbfu{\check{\bfu}}
\def\cbfv{\check{\bfv}}

\def\Weff{\overline{W}}
\def\Ueff{\overline{U}}
%\def\Leff{\overline{\bfL}}
\def\Lreff{\overline{\Lr}}

\def\r{^{(r)}}
\def\p{\partial}

%Definitions: journals
\def\AMPA{{\it Ann. Mat. Pura Appl.\ }}
\def\AMR{{\it Appl. Mech. Rev.\ }}
\def\ARMA{Arch. Rat. Mech. Analysis\ }
\def\ASCEEM{{\it ASCE J. Eng. Mech.\ }}
\def\AAM{{\it Adv. Appl. Mech.\ }}
\def\CMAME {{\it Comput. Meth. Appl. Mech. Engrg.\ }}
\def\CS {{\it Comput. Struct.\ }}
\def\CRAS{C.R. Mecanique\ }
\def\EFM{{\it Eng. Fracture Mechanics\ }}
\def\EJMA{{\it Eur.~J.~Mechanics-A/Solids\ }}
\def\IJES{{\it Int. J. Eng. Sci.\ }}
\def\IJMS{{\it Int. J. Mech. Sci.\ }}
\def\IJNAMG{{\it Int. J. Numer. Anal. Meth. Geomech.\ }}
\def\IJP{{\it Int. J. Plasticity\ }}
\def\IJSS{Int. J. Solids Struct.\ }
\def\IngA{{\it Ing. Archiv\ }}
\def\JAM{{\it J. Appl. Mech.\ }}
\def\JAP{{\it J. Appl. Phys.\ }}
\def\JE{J. Elast.\ }
\def\JM{{\it J. de M\'ecanique\ }}
\def\JMPS{J. Mech. Phys. Solids\ }
\def\JPD{{\it J. Phys. D: Appl. Phys.\ }}
\def\MA{{\it Macromolecules\ }}
\def\MMS{{\it Math. Mech. Solids\ }}
\def\MOM{{\it Mech. Materials\ }}
\def\MRSSP{{\it Mat. Res. Soc. Symp. Proc.\ }}
\def\MTT{{\it Mekh. Tverd. Tela.\ }}
\def\MPCPS{{\it Math. Proc. Cambridge Phil. Soc.\ }}
\def\PRSLA{Proc. R. Soc. Lond. A\ }
\def\QAM{{\it Quart. Appl. Math.\ }}
\def\QJMAM{{\it Quart. J. Mech. Appl. Math.\ }}
\def\RAN{{\it Rend. Acc. Naz.\ }}
\def\RCT{{\it Rubb. Chem. Technol.\ }}

\def\doublelow#1{\,\vtop{\ialign{\hfil$##$\hfil\crcr
                    \mathstrut #1 \crcr}}\,}
\def\psiEq{\psi^{\rm Eq}}
\def\psiNEq{\psi^{\rm NEq}}
\newcommand{\croch}[1]{{\left[ #1 \right]}}
\newcommand{\accol}[1]{{\left\{ #1 \right\}}}
\newcommand{\del}[2]{\frac{\partial #1}{\partial #2}}
\newcommand{\ddel}[2]{\frac{\partial^2 #1}{\partial {#2}^2}}
\newcommand{\bigdel}[2]{\dfrac{\partial #1}{\partial #2}}
\newcommand{\lr}[1]{\left(#1\right)}
\long\def\symbolfootnote[#1]#2{\begingroup%
\def\thefootnote{\fnsymbol{footnote}}\footnote[#1]{#2}\endgroup}
\setlength{\parindent}{0pt}
% \usepackage{amsmath,amsfonts,amssymb}
% \usepackage[dvipsnames]{xcolor}
\title{Calculations: J. Elasticity -- Bubbles}
\author{Bhavesh}

\begin{document}

\frame{\titlepage}

\begin{frame}
\frametitle{Model}\vspace*{-2ex}
\begin{block}{Potentials}
    \begin{equation*}
        \mathrm{\Psi}(\boldsymbol{\mathsf{F}},\boldsymbol{\mathsf{F}}^v)=\mathrm{\Psi}^{{\rm Eq}}(\boldsymbol{\mathsf{F}})
        +J^v\mathrm{\Psi}^{{\rm NEq}}\left(\boldsymbol{\mathsf{F}}{\boldsymbol{\mathsf{F}}^v}^{-1}, \mathsf{F}^v\right) \label{psi_effective}
    \end{equation*}
\begin{equation*}
    \myPhi(\boldsymbol{\mathsf{F}},\boldsymbol{\mathsf{F}}^v,\dot{\boldsymbol{\mathsf{F}}}^v)=\dfrac{1}{2}J^v\dot{\boldsymbol{\mathsf{F}}}^v{\boldsymbol{\mathsf{F}}^v}^{-1}\cdot\left[
    \boldsymbol{\mathsf{M}}{\boldsymbol{\mathsf{F}}}^v{\boldsymbol{\mathsf{F}}^v}^{-1}\right]\quad \textsf{with}\quad \boldsymbol{\mathsf{M}}=2 \mathsf{m}_{\mathcal{K}}\Ktan+3\mathsf{m}_{\mathcal{J}}\Jtan,  \label{phi_effective}
\end{equation*}
\end{block}
\begin{block}{Evolution equation}
    \begin{equation*}
        \del{\Psi}{\mathsf{F}^v}
        +
        \del{\myPhi}{\dot{\mathsf{F}}^v} = {\bf 0} \Longleftrightarrow 
        \del{\myPhi}{\dot{\mathsf{F}}^v} = -\del{\myPsi}{\mathsf{F}^v}
    \end{equation*}
\end{block}
Let, $\Gamma^v = \dot{\mathsf{F}}^v{\mathsf{F}^v}^{-1}$ and apply chain rule!
\begin{block}{}
    \begin{equation*}
        \del{\myPhi}{\dot{\myF}^v}
        =
        \del{\myPhi}{\Gamma^v}\cdot\del{\Gamma^v}{\dot{\myF}^v}\Longleftrightarrow
        \del{\myPhi}{\Gamma^v_{mn}}\del{\Gamma^v_{mn}}{\dot{\myF}^v_{ij}}
    \end{equation*}
\end{block}
\end{frame}
\begin{frame}{Model..}
    \begin{block}{}
        \footnotesize\begin{equation*}
            \del{\Gamma^v_{mn}}{\dot{\myF}^v_{ij}} = \delta_{mi}\myF^{v^{-1}}_{jn}\quad\text{and}\quad\del{\myPhi}{\Gamma^v_{mn}} = \myM_{mnrs}\Gamma^v_{rs}
        \end{equation*}
        \begin{equation*}
            \myM_{mnrs}\Gamma^v_{rs} =  m_{\mathcal{K}}\lr{
                \Gamma^v_{mn} + \Gamma^v_{nm}
            } + \left(m_\mathcal{J} - \frac{2}{3}m_\mathcal{K}\right)\Gamma^v_{rr}\delta_{mn}
        \end{equation*}
        \begin{align*}
            \Gamma^v_{rr} \Longleftrightarrow \text{tr}\lr{
                \Gamma^v
            } = \frac{1}{2}\dot{\bf C}^v\cdot{\bf C}^{v^{-1}}\qquad \text{where}\quad {\bf C}^v = {{\bf F}^v}^T {\bf F}^v
        \end{align*}
    \end{block}
    \begin{block}{Derivative of free energy}
        \footnotesize\begin{align*}
            \del{\myPsi}{\myF^v}
            & =
            \del{\myPsi}{I^e_1}\del{I^e_1}{\myF^v} + 
            \del{\myPsi}{J^v}\del{J^v}{\myF^v} \\
            \del{I^e_1}{\myF^v} & = -2 {\myF^v}^{-T}\myC{\myF^v}^{-1}{\myF^v}^{-T}\quad\text{and}\quad
            \del{J^v}{\myF^v} = J^v{\myF^v}^{-T} \\
            \del{\myPsi}{I^e_1} & = J^v\del{\myPsi^{\rm NEq}}{I^e_1};\qquad
            \del{\myPsi}{J^v} = \left( \myPsi^{\rm NEq} + J^v\del{\myPsi^{\rm NEq}}{J^v}\right)
        \end{align*}
    \end{block}
    \end{frame}

    \begin{frame}{Model....}
        \begin{block}{Free Energy}
            \footnotesize\begin{align*}
                \del{\myPsi}{\myF^v}=
                -2J^v \del{\myPsi^{\rm NEq}}{I^e_1}{\myF^v}^{-T}\myC{\myF^v}^{-1}{\myF^v}^{-T} +
                \left({\del{\myPsi}{J^v}}\right) J^v{\myF^v}^{-T}
            \end{align*}
        \end{block}
        \begin{block}{Dissipation potential}
            \footnotesize\begin{align*}
                \del{\myPhi}{\dot{\mathsf{\bf F}}^v}
            =
            J^vm_\mathcal{K}
            \left(
                \dot{\mathsf{F}}^v {\mathsf{F}^v}^{-1} {\mathsf{F}^v}^{-T}
                +
                {\mathsf{F}^v}^{-T} {\dot{\mathsf{F}}}^{v^T} {\mathsf{F}^v}^{-T}
            \right) +J^v
            \lr{m_\mathcal{J}
            - \frac{2}{3}m_\mathcal{K}
            }\text{tr}\lr{\Gamma^v}{\mathsf{F}^v}^{-T} 
            \end{align*}
        \end{block}
        \begin{block}{Combined}
            \footnotesize
            \begin{align*}
                & 2J^v \del{\myPsi^{\rm NEq}}{I^e_1}{\myF^v}^{-T}\myC{\myF^v}^{-1}{\myF^v}^{-T} -
                J^v\left({\del{\myPsi}{J^v}}\right) {\myF^v}^{-T}\\ & = J^vm_\mathcal{K}
                \left(
                    \dot{\mathsf{F}}^v {\mathsf{F}^v}^{-1} {\mathsf{F}^v}^{-T}
                    +
                    {\mathsf{F}^v}^{-T} {\dot{\mathsf{F}}}^{v^T} {\mathsf{F}^v}^{-T}
                \right)+J^v
                \lr{m_\mathcal{J}
                - \frac{2}{3}m_\mathcal{K}
                }\text{tr}\lr{\Gamma^v}{\mathsf{F}^v}^{-T}
            \end{align*}
        \end{block}
    \end{frame}
    \begin{frame}
        \begin{block}{\footnotesize Pre- and post-multiply by ${\myF^v}^{T}$ and cancel $J^v$}
            \footnotesize\begin{align*}
                2\del{\myPsi^{\rm NEq}}{I^e_1}\myC{\myF^v}^{-1} -
                \left({\del{\myPsi}{J^v}}\right) {\myF^v}^{T} = m_\mathcal{K}
                \left(
                    {\myF^v}^T\dot{\mathsf{F}}^v {\mathsf{F}^v}^{-1}
                    +
                    {\dot{\mathsf{F}}}^{v^T}
                \right)+
                \lr{m_\mathcal{J}
                - \frac{2}{3}m_\mathcal{K}
                }\text{tr}\lr{\Gamma^v}{\mathsf{F}^v}^{T}
            \end{align*}
        \end{block}
        \begin{block}{\footnotesize Post-multiply by ${\myF^v}$}
            \footnotesize\begin{align*}
                2 \del{\myPsi^{\rm NEq}}{I^e_1}\myC -
                \left({\del{\myPsi}{J^v}}\right) {\myC^v}= m_\mathcal{K}
                \left(
                    {\myF^v}^T\dot{\mathsf{F}}^v
                    +
                    {\dot{\mathsf{F}}}^{v^T} {\mathsf{F}}^v
                \right)
                +
                \lr{m_\mathcal{J}
                - \frac{2}{3}m_\mathcal{K}
                }\text{tr}\lr{\Gamma^v}\myC^v
            \end{align*}
        \end{block}
        \begin{block}{\footnotesize Post multiply by ${\myC^v}^{-1}$}
            \footnotesize\begin{align*}
                2 \del{\myPsi^{\rm NEq}}{I^e_1}\myC{\myC^v}^{-1} -
                \left({\del{\myPsi}{J^v}}\right){\bf I}
                = m_\mathcal{K}
                \left(
                    \dot{\myC}^v{{\myC}^v}^{-1}
                \right) 
                + \frac{1}{2}
                \lr{m_\mathcal{J}
                - \frac{2}{3}m_\mathcal{K}
                }\dot{\myC}^v\cdot{{\myC}^v}^{-1}\,{\bf I}
            \end{align*}
        \end{block}
        \begin{block}{\footnotesize Finally, take the trace}
            \footnotesize\begin{align*}
                2 \del{\myPsi^{\rm NEq}}{I^e_1}\myC\cdot{\myC^v}^{-1} -
                3\left({\del{\myPsi}{J^v}}\right) 
                = m_\mathcal{K}
                \left(
                    \dot{\myC}^v\cdot{{\myC}^v}^{-1}
                \right) 
                + \frac{3}{2}
                \lr{m_\mathcal{J}
                - \frac{2}{3}m_\mathcal{K}
                }\dot{\myC}^v\cdot{{\myC}^v}^{-1}
            \end{align*}
        \end{block}
    \end{frame}
    \begin{frame}
        \begin{block}{Therefore,}
            \footnotesize\color{blue}\begin{align*}
                \dot{\myC}^v\cdot{\myC^v}^{-1}
                =
                \frac{2}{m_\mathcal{J}}\lrB{
                    \frac{2}{3}\del{\myPsi^{\rm NEq}}{I^e_1}{\myC}\cdot{\myC^v}^{-1}
                -
                \left({\del{\myPsi}{J^v}}\right)
                }
            \end{align*}
        \end{block}
        \begin{block}{Substituting back,}
            \footnotesize\begin{align*}
                & 2 \del{\myPsi^{\rm NEq}}{I^e_1}\myC{\myC^v}^{-1} -
                \left({\del{\myPsi}{J^v}}\right) {\bf I} = m_\mathcal{K}
                \left(
                    \dot{\myC}^v{\myC^v}^{-1}
                \right)
                \\ &+
                \frac{1}{m_\mathcal{J}}
                \lr{m_\mathcal{J}
                - \frac{2}{3}m_\mathcal{K}
                }\lrB{
                    \frac{2}{3}\del{\myPsi^{\rm NEq}}{I^e_1}{\myC}\cdot{\myC^v}^{-1}
                -
                \left({\del{\myPsi}{J^v}}\right)
                }{\bf I}
            \end{align*}
        \end{block}
        \begin{block}{where,}
            \footnotesize\begin{align*}
                2 \del{\myPsi^{\rm NEq}}{I^e_1} = \overline{\nu} 
                =
                \frac{3(1-f^v)}{(3+2f^v)}\nu_{m}
                =
                \frac{3(1 - f_0)}{(5 J^v + 2f_0  - 2)}\nu_m
            \end{align*}
        \end{block}
    \end{frame}
    \begin{frame}
        \begin{block}{Substituting this in the above equation,}
            \footnotesize
            \begin{align*}
                \overline{\nu}\myC{\myC^v}^{-1} -
                \left({\del{\myPsi}{J^v}}\right) {\bf I} = m_\mathcal{K}
                \left(
                    \dot{\myC}^v{\myC^v}^{-1}
                \right)
                +
                \frac{1}{m_\mathcal{J}}
                \lr{m_\mathcal{J}
                - \frac{2}{3}m_\mathcal{K}
                }\lrB{
                    \frac{1}{3}\overline{\nu}{\myC}\cdot{\myC^v}^{-1}
                -
                \left({\del{\myPsi}{J^v}}\right)
                }{\bf I}
            \end{align*}
        \end{block}
        \begin{block}{Post multiply by $\myC^v$}
            \footnotesize
            \begin{align*}
                & \overline{\nu}\myC -
                \left({\del{\myPsi}{J^v}}\right) {\myC}^v = m_\mathcal{K}
                \left(
                    \dot{\myC}^v
                \right)
                +
                \frac{1}{m_\mathcal{J}}
                \lr{m_\mathcal{J}
                - \frac{2}{3}m_\mathcal{K}
                }\lrB{
                    \frac{\overline{\nu}}{3}{\myC}\cdot{\myC^v}^{-1}
                -
                \left({\del{\myPsi}{J^v}}\right)
                }{\myC}^v\\
                \implies &
                \overline{\nu}\myC -
                \left({\del{\myPsi}{J^v}}\right) {\myC}^v
                 = m_\mathcal{K}
                \left(
                    \dot{\myC}^v
                \right)
                +
                \lr{1
                - \frac{2}{3}\frac{m_\mathcal{K}}{m_\mathcal{J}}
                }\lrB{
                    \frac{\overline{\nu}}{3}{\myC}\cdot{\myC^v}^{-1}
                -
                \left({\del{\myPsi}{J^v}}\right)
                }{\myC}^v
            \end{align*}
        \end{block}
    \end{frame}
    \begin{frame}
        \begin{block}{\footnotesize Simplifying a bit more...}
           \footnotesize\begin{align*}
            & \overline{\nu}\myC -
            \left({\del{\myPsi}{J^v}}\right) {\myC}^v
             = m_\mathcal{K}
            \left(
                \dot{\myC}^v
            \right)
            +
            \lr{1
            - \frac{2}{3}\frac{m_\mathcal{K}}{m_\mathcal{J}}
            }\lrB{
                \frac{\overline{\nu}}{3}{\myC}\cdot{\myC^v}^{-1}}{\myC}^v
            \\ 
            &-\lr{1-\frac{2}{3}\frac{m_\mathcal{K}}{m_\mathcal{J}}}\lrB{    
           {\del{\myPsi}{J^v}}
            }\myC^v\\
            \implies & \overline{\nu}\myC = m_\mathcal{K}\dot{\myC}^v+\lr{1
            - \frac{2}{3}\frac{m_\mathcal{K}}{m_\mathcal{J}}
            }\lrB{
                \frac{\overline{\nu}}{3}{\myC}\cdot{\myC^v}^{-1}}{\myC}^v
                +
                \frac{2m_\mathcal{K}}{3m_\mathcal{J}}\lrB{\color{red}    
                   {\del{\myPsi}{J^v}}
                    }\myC^v
           \end{align*}
        \end{block}
        \begin{block}{\footnotesize Coefficient with $\myC\cdot{\myC^v}^{-1}$}
            \footnotesize\begin{align*}
                &\frac{\overline{\nu}}{3}+\lr{\frac{2m_\mathcal{K}}{3m_\mathcal{J}}}\lrB{
                    -\frac{\overline{\nu}}{3}
                    +\dots\text{\color{red}Coming from the second term}
                }\\
                \implies & \frac{\overline{\nu}}{3}+\lr{\frac{2m_\mathcal{K}}{3m_\mathcal{J}}}
                \lrB{
                    -\frac{(f_0-1)\left(f_0-5 J^v-1\right)}{\left(2 f_0+5 J^v-2\right){}^2}
                    \nu _m } = \alpha
            \end{align*}
        \end{block}
    \end{frame}
    \begin{frame}
        \begin{block}{\footnotesize Remaining RHS}
            \tiny\begin{align*}
                \frac{1}{4} \nu _m 
                \left(8 J^{2/3} {J^v}^{1/3} 
                    \left(-\frac{15 (f_0-1)}{\left(2 f_0+5 {J^v}-2\right){}^2}-\frac{1}{J}\right)
                -\frac{75 J^{2/3} {J^v}^{4/3}}{\left(2 f_0+5 {J^v}-2\right){}^2}
                +\frac{4 \left(f_0+2 {J^v}-J-1\right)}{{\lr{J + f_0 - 1}}^{1/3} \left(f_0+{J^v}-1\right){}^{2/3}}+7 \left(\frac{J}{{J^v}}\right){}^{2/3}\right)\\ = \Theta\quad\text{say}
            \end{align*}
        \end{block}
        % \begin{block}{\footnotesize Remaining RHS, sans $\nu_m$ (with a minus sign!)}
        %     \footnotesize\begin{align*}
        %         J^{2/3} {J^v}^{1/3} &\left(\frac{2}{J}-\frac{5 (f_0-1)}{\left(2 f_0+5 {J^v}-2\right){}^2}\right)
        %         -\frac{25 J^{2/3} {J^v}^{4/3}}{\left(2 f_0+5 {J^v}-2\right){}^2}\\
        %         &+\frac{-f_0-2 {J^v}+J+1}{\lr{J+f_0 - 1}^{1/3} \left(f_0+{J^v}-1\right){}^{2/3}}
        %         -\frac{7 (f_0-1)^2 \left(\frac{J}{{J^v}}\right){}^{2/3}}{\left(2 f_0+5 {J^v}-2\right){}^2} = -\Theta (say!)
        %     \end{align*}
        % \end{block}
        \begin{block}{\footnotesize Simplifying and rewriting}
            \footnotesize\begin{align*}
                &m_\mathcal{K}\lr{\dot{\myC}^v}
                =\overline{\nu}\myC - \frac{2m_\mathcal{K}}{3m_\mathcal{J}}\Theta\myC^v
                -
                \lr{\alpha\myC\cdot{\myC^v}^{-1}}\myC^v\\
                \implies &
                {\dot{\myC}^v} = \frac{\overline{\nu}}{m_\mathcal{K}}\myC
                -
                \frac{2}{3m_\mathcal{J}}\Theta\myC^v-
                \lr{
                    \frac{\alpha}{m_\mathcal{K}}\myC\cdot{\myC^v}^{-1}
                }\myC^v
            \end{align*}
        \end{block}
        \begin{block}{\footnotesize Simplifying $\alpha/m_\mathcal{K}$}
            \footnotesize\begin{align*}
                \frac{\alpha}{m_\mathcal{K}}=\frac{\overline{\nu}}{3m_\mathcal{K}}
                +
                \lr{\frac{2}{3m_\mathcal{J}}} \lrB{
                    \frac{\lr{1-f_0}\left(f_0-5 J^v-1\right)}{\left(2 f_0+5 J^v-2\right){}^2}
                    \nu _m }
            \end{align*}
        \end{block}
    \end{frame}
    \begin{frame}
        \begin{block}{\footnotesize Simplifying $\alpha/m_\mathcal{K}$...}
            \footnotesize\begin{align*}
                \frac{\alpha}{m_\mathcal{K}}
                &=\frac{\overline{\nu}}{3m_\mathcal{K}}
                +
                \lr{\frac{2}{3m_\mathcal{J}}} \lrB{
                    \frac{\lr{1-f_0}\left(f_0-5 J^v-1\right)}{\left(2 f_0+5 J^v-2\right){}^2}
                    \nu _m }\\
                    & = \frac{\nu_m}{3\eta_m} + \frac{2\nu_m}{3\eta_m}\times\lr{\frac{\eta_m}{m_\mathcal{J}}}\lrB{
                        \frac{\lr{1-f_0}\left(f_0-5 J^v-1\right)}{\left(2 f_0+5 J^v-2\right){}^2}
                       }\\
                       \implies  -\frac{\alpha}{m_\mathcal{K}}
                       &=\frac{2\nu_m}{3\eta_m}\lrB{
                           -\frac{1}{2} + \frac{\lr{1-f_0}\left(-f_0+5 J^v+1\right)}{\left(2 f_0+5 J^v-2\right){}^2}\frac{\eta_m}{m_\mathcal{J}}
                       }\quad\text{\color{blue}Matched!!}
            \end{align*}
        \end{block}
        \begin{block}{\footnotesize Simplifying terms in $\Theta$...}
            \footnotesize\begin{align*}
                -\frac{2}{3m_\mathcal{J}}\Theta_{(3)} 
                = 
                \frac{2\nu_m}{3\eta_m}\times \frac{\eta_m}{m_\mathcal{J}}
                \frac{J -f_0 - 2 {J^v} + 1}{\lr{J+f_0 - 1}^{1/3} \left(f_0+{J^v}-1\right){}^{2/3}}\qquad \text{\color{blue}Matched!!}
            \end{align*}
            % \footnotesize\begin{align*}
            %     \frac{2}{3m_\mathcal{J}}\Theta_{(1)_1} 
            %     = 
            %     \frac{2 \sqrt[3]{J^v} \nu _m}{\sqrt[3]{J}}
            %     =
            %     2\lr{\frac{J^v}{J}}^{1/3}
            %     \qquad \text{\color{blue}Matched!!}
            % \end{align*}
        \end{block}
        \begin{block}{\footnotesize Now combining $\Theta_{(1)_1}$ and $\Theta_{(2)}$}
            \footnotesize\begin{align*}
                -\frac{15J^{2/3} {J^v}^{1/3}
                \left(
                    5 {J^v} + 8 f_0 - 8
                \right) \nu _m }{4\left(5 {J^v} + 2 f_0 -2\right){}^2}
            \end{align*}
        \end{block}
        % \begin{block}{Now evaluating: $\left({\del{\myPsi}{J^v}}\right)$}
        %     \footnotesize\begin{align*}
        %         =\frac{J^v\nu_m}{4}
        %         \left(
        %             \frac{12 (f_0-1)^2 
        %             \left(
        %                 3 \left(\frac{J}{J^v}\right)^{2/3}
        %                 -
        %                 I^e_1
        %             \right)}{\left(2 f_0+5 J^v-2\right)^2}
        %             +
        %             \frac{150 {J^v}^2 \left(\frac{J}{J^v}\right)^{2/3}}{\left(2 f_0+5 J^v-2\right)^2}
        %             +\right. \\
        %             \left.
        %             \frac{4 \left(f_0+2 J^v-J-1\right)}{J^v \left(\frac{f_0-1}{J^v}+1\right)^{2/3} \sqrt[3]{\frac{f_0+J-1}{J^v}}}
        %             +
        %             \frac{4 J \left(\frac{15 (f_0-1)}{\left(2 f_0+5 J^v-2\right){}^2}-\frac{2}{J}\right)}{\sqrt[3]{\frac{J}{J^v}}}-2 \left(\frac{J}{J^v}\right)^{2/3}
        %         \right)
        %     \end{align*}
        % \end{block}
        % \begin{block}{But remember}
        %     \footnotesize\begin{align*}
        %         I^e_1 = \myC\cdot{\myC^v}^{-1}\qquad\text{and this is there {\color{red} in this term!!}}
        %     \end{align*}
        %     also the coefficient of $I^e_1$ in the above term is
        %     \begin{align*}
        %         -\frac{3 (f_0-1)^2 \nu _m}{\left(2 f_0+5 J^v-2\right)^2}
        %     \end{align*}
        % \end{block}
    \end{frame}
    \begin{frame}
        \begin{block}{\footnotesize which leaves only $\Theta_{(1)_2}$ and $\Theta_{(4)}$ (remember to take the negative sign) }
            \footnotesize\begin{align*}
                \frac{ \nu _m}{4} \left(7 \left(\frac{J}{{J^v}}\right)^{2/3}
                -
                8\lr{\frac{J^v}{J}}^{1/3}\right)
            \end{align*}
        \end{block}
        \begin{block}{\footnotesize Putting it together}
            \footnotesize\begin{align*}
                \dot{\myC}^v
               & =
                \frac{\nu_m}{\eta_m}\myC
                +
                \frac{2\nu_m}{3\eta_m}\left[
                    \frac{\eta_m}{m_\mathcal{J}}\left(
                        \frac{8J^v-7J}{4{J^v}^{2/3}J^{1/3}}
                        +\frac{J - 2 {J^v} - f_0 + 1}{\lr{J+f_0 - 1}^{1/3} \left({J^v}+f_0-1\right)^{2/3}}+\right.\right. \\ &\left.\left.
                        \, \frac{15J^{2/3} {J^v}^{1/3}
                        \left(
                            5 {J^v} + 8 f_0 - 8
                        \right)}{4\left(5 {J^v} + 2 f_0 -2\right)^2}
                    \right)
                    + \lr{
                        \frac{\lr{1-f_0}\left(5 J^v -f_0 + 1\right)}{\left(2 f_0+5 J^v-2\right)^2}\frac{\eta_m}{m_\mathcal{J}}-\frac{1}{2}
                    }\myC\cdot{\myC^v}^{-1}
                \right]\myC^v
            \end{align*}
        The evolution equation is the same as Equation (57) in the paper. This can be checked quickly by subtracting the differing terms (which equates to 0!). The same goes for the first Piola-Kirchhoff stress tensor.
        \end{block}
    \end{frame}
    % \begin{frame}
    %     \begin{block}{Looking term that multiplies $\myC\cdot{\myC^v}^{-1}$ (RHS!!)}
    %         \begin{align*}
    %             \left(\frac{\overline{\nu}}{3} + \frac{3 (f_0-1)^2 \nu _m}{\left(2 f_0+5 J^v-2\right)^2}
    %             \right) = 
    %             \frac{(f_0-1)  \left(f_0-5 J^v-1\right)}{\left(2 f_0+5 J^v-2\right)^2}\nu _m
    %         \end{align*}
    %         This further multiplies $\frac{1}{m_\mathcal{K}}\lr{1
    %         - \frac{2}{3}\frac{m_\mathcal{K}}{m_\mathcal{J}}
    %         }$
    %         which can be expanded as
    %         \begin{align*}
    %             \frac{1}{m_\mathcal{K}} \lrB{\frac{(f_0-1)  \left(f_0-5 J^v-1\right)}{\left(2 f_0+5 J^v-2\right)^2}\nu _m}
    %             =
    %             -\frac{\nu _m \left(f_0-5 J^v-1\right)}{3 \eta _m \left(2 f_0+5 J^v-2\right)}
    %         \end{align*}
    %         and
    %         \begin{align*}
    %             -\frac{2}{3m_\mathcal{J}} \lrB{\frac{(f_0-1)  \left(f_0-5 J^v-1\right)}{\left(2 f_0+5 J^v-2\right)^2}\nu _m}
    %             =
    %             -\frac{2 (f_0-1)  \left(f_0-5 J^v-1\right)}{3 \left(2 f_0+5 J^v-2\right)^2}\frac{\nu_m}{m_\mathcal{J}}
    %         \end{align*}
    %     \end{block}
    % \end{frame}
    % \begin{frame}
    %     \begin{block}{Factoring out terms to match evolution equation (57)}
    %         Term with $\myC\cdot{\myC^v}^{-1}$ after taking to the LHS
    %         \footnotesize\begin{align*}
    %             \frac{2\nu_m}{3\eta_m}\lrB{
    %                 \frac{(1-f_0)  \left(-f_0+5 J^v+1\right)}{\left(2 f_0+5 J^v-2\right)^2}\frac{\eta_m}{m_\mathcal{J}}
    %                 -\frac{1}{2} \frac{\left(-f_0 + 5 J^v + 1\right)}{\left(2 f_0+5 J^v-2\right)}
    %             }
    %         \end{align*}
    %         {
    %             \color{blue} Note that the first term is matching but the second isn't.
    %         }
    %     \end{block}
    %     \begin{block}{Another term that multiplies $\myC^v$ directly (RHS)}
    %         \footnotesize\begin{align*}
    %             \frac{\nu _m \left(f_0+2 J^v-J-1\right)}{
    %                 \left({J^v + f_0-1}\right)^{2/3}
    %                 \lr{J+f_0 - 1}^{1/3}}
    %         \end{align*}
    %         Then multiplied by $\frac{2}{3m_\mathcal{J}}$ (two minus signs on the RHS). After taking to LHS, becomes
    %         \begin{align*}
    %             \frac{2}{3m_\mathcal{J}}\lrB{
    %                 \frac{\nu _m \left(-f_0 - 2 J^v + J + 1\right)}{
    %                 \left({J^v + f_0-1}\right)^{2/3}
    %                 \lr{J+f_0 - 1}^{1/3}}
    %             }
    %             =
    %             \frac{2\nu_m}{3\eta_m}
    %             \lrB{\lr{\frac{\eta_m}{m_\mathcal{J}}}
    %                 \frac{\left(J- 2 J^v -f_0  + 1 \right)}{
    %                 \left({J^v + f_0-1}\right)^{2/3}
    %                 \lr{J+f_0 - 1}^{1/3}}
    %             }
    %         \end{align*}
    %     \end{block}
    % \end{frame}
    % \begin{frame}
    %     \begin{block}{Final evolution equation}
    %         \footnotesize\begin{align*}
    %             \dot{\myC}^{v} = \frac{5J^v + 2(f_0 - 1)}{J^v\eta_m\lr{3+2f_0}}{\myC}
    %             &+
    %             \frac{J^v + f_0 - 1}{2\eta_m(1-f_0)J^v}J^e\del{\myPsi}{J^e}{\bf \myC^v}\\
    %             &-  \lr{\frac{(7J^v+f_0 - 1)\nu_m}{6\eta_mJ^v(3+2f_0)}\myC\cdot{\myC^v}^{-1}}
    %             {\bf \myC^v}
    %         \end{align*}
    %     \end{block}
    % \end{frame}
\end{document}